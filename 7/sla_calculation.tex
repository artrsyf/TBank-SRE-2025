\documentclass{article}
\usepackage[utf8]{inputenc}
\usepackage[russian]{babel}
\usepackage{amsmath}
\usepackage{amssymb}
\usepackage[a4paper, left=20mm, right=20mm, top=20mm, bottom=20mm]{geometry}

\begin{document}

\section*{Формула подсчёта SLA сервиса}

\vspace{0.5cm}

\textbf{Обозначения и параметры:}

\begin{itemize}
    \item \( U \) - \textbf{Средний процент доступности} (availability), измеряемый как доля времени, когда сервис отвечает на проверки работоспособности (пробы, \textit{probe\_up}):
    \[
    U = \text{среднее значение } \left( \text{probe\_up} \right) \times 100\%
    \]
    Здесь \textit{probe\_up} - бинарный индикатор, который равен 1, если сервис доступен и отвечает на проверку, и 0, если нет. Среднее берётся за последний период (например, 30 минут).
    
    \item \( S \) - \textbf{Процент успешных запросов} (success ratio):
    \[
    S = \left( 1 - \frac{\sum \text{ошибочные запросы}}{\sum \text{всех запросов}} \right) \times 100\%
    \]
    Этот показатель отражает долю корректно обработанных запросов к сервису за выбранный период (например, последние 5 минут). Ошибочные запросы учитывают любые виды ошибок на уровне обработки.
    
    \item \( \mathrm{avgLatency} \) - \textbf{99-процентный перцентиль времени отклика} сервиса, то есть значение времени ответа, которое не превышает 99\% всех запросов.
    
    \item \( \mathrm{SLO_{latency}} \) - \textbf{Целевой порог времени отклика} (Service Level Objective по латентности), установленный для сервиса. Например, 0.5 секунды.
    
    \item \( L \) - \textbf{Показатель соответствия времени отклика SLO} (latency SLO ratio), выраженный в процентах:
    \[
    L = 
    \begin{cases}
    100, & \text{если } \mathrm{avgLatency} \leq \mathrm{SLO_{latency}} \\
    100 \times \frac{\mathrm{SLO_{latency}}}{\mathrm{avgLatency}}, & \text{если } \mathrm{avgLatency} > \mathrm{SLO_{latency}}
    \end{cases}
    \]
    То есть, если сервис отвечает быстрее или ровно в целевое время, показатель равен 100\%. Если время ответа выше цели, показатель снижается пропорционально.
\end{itemize}

\vspace{0.5cm}

\textbf{Итоговая формула SLA:}

Общий показатель SLA для сервиса рассчитывается как среднее арифметическое трёх вышеописанных метрик:

\[
\boxed{
\mathrm{SLA} = \frac{U + S + L}{3}
}
\]

где

\begin{itemize}
    \item \( U \) - средний процент доступности,
    \item \( S \) - процент успешных запросов,
    \item \( L \) - процент соответствия целевому времени отклика.
\end{itemize}

\vspace{0.5cm}

\end{document}
